\documentclass[a4paper]{article}
\usepackage[margin=1in]{geometry}
\usepackage[T1]{fontenc}
\usepackage[utf8]{inputenc}
\usepackage{newcent}
\usepackage{helvet}
\usepackage{graphicx}
\usepackage{dirtree}
\usepackage[pdftex, pdfborder={0 0 0}]{hyperref}
\frenchspacing

\usepackage{txfonts} % For \varheartsuit and \vardiamondsuit
\usepackage[usenames,dvipsnames]{color} % dvipsnames necessary to made PDFLaTeX work.
\usepackage{enumerate}
\usepackage{listliketab}
\usepackage{latexsym} % \Box
\usepackage{pbox} % \Box
\usepackage{parskip} % line between paragraphs

% suits

%%% Colors

\newcommand{\BC}{\textcolor{OliveGreen}{$\clubsuit$}}
\newcommand{\BD}{\textcolor{RedOrange}{$\vardiamondsuit$}}
\newcommand{\BH}{\textcolor{Red}{$\varheartsuit${}}}
\newcommand{\BS}{\textcolor{Blue}{$\spadesuit${}}}

%suits for pdf-friendly titles
\newcommand{\pdfc}{\texorpdfstring{\BC{}}{C}}
\newcommand{\pdfd}{\texorpdfstring{\BD{}}{D}}
\newcommand{\pdfh}{\texorpdfstring{\BH{}}{H}}
\newcommand{\pdfs}{\texorpdfstring{\BS{}}{S}}

\newenvironment{bidtable}
{\begin{tabbing}

xxxxxx\=xxxxxx\=xxxxxx\=xxxxxx\=xxxxxx\=xxxxxx\=xxxxxx\=xxxxxx\=xxxxxx\=xxxxxx\=\kill}
{\end{tabbing} }%

\newenvironment{bidding}%
{\begin{tabbing}
xxxxxx\=xxxxxx\=xxxxxx\=xxxxxx \kill
}{\end{tabbing} }%end bidding


% writing hands
\newcommand{\cards}[1]{\textsf{#1}}
\newcommand{\spades}[1]{\BS\cards{#1}}
\newcommand{\hearts}[1]{\BH\cards{#1}}
\newcommand{\diamonds}[1]{\BD\cards{#1}}
\newcommand{\clubs}[1]{\BC\cards{#1}}
\newcommand{\void}{--}
\newcommand{\hand}[4]{\spades{#1}\ \hearts{#2}\ \diamonds{#3} \clubs{#4}}
\newcommand{\vhand}[4]{\spades{#1}\\\hearts{#2}\\\diamonds{#3}\\\clubs{#4}}

% The \Box should always appear the same distance from the left margin
\newcommand\onesuit[4]%
{%
  \begin{center}%
    \begin{tabular}{>{\hfill}p{3cm}cp{3cm}}
                & \cards{#2} \\%
      \cards{#1}& $\Box$    & \cards{#3} \\%
                & \cards{#4} %
    \end{tabular}
  \end{center}%
}

% A special command if the south hand is not shown to avoid whitespace
\newcommand\onesuitenw[3]%
{%
  \begin{center}%
    \begin{tabular}{>{\hfill}p{3cm}cp{3cm}}%
                & \cards{#2} \\%
      \cards{#1}& $\Box$    & \cards{#3}%
    \end{tabular}%
  \end{center}%
}

\newcommand\dealdiagram[5]%
{%
  \begin{center}%
    \begin{tabular}{>{\hfill}p{3cm}cp{3cm}}
      \pbox{20cm}{\small #5}& \pbox{20cm}{#2} \\%
      \pbox{20cm}{#1}& $\Box$    & \pbox{20cm}{#3} \\%
                & \pbox{20cm}{#4} %
    \end{tabular}
  \end{center}%
}

\newcommand\dealdiagramenw[4]%
{%
  \begin{center}%
    \begin{tabular}{>{\hfill}p{3cm}cp{3cm}}
      \pbox{20cm}{\small #4}& \pbox{20cm}{#2} \\%
      \pbox{20cm}{#1}& $\Box$    & \pbox{20cm}{#3} \\%
    \end{tabular}
  \end{center}%
}

\newcommand\dealdiagramew[2]%
{%
  \begin{center}%
    \begin{tabular}{>{\hfill}p{3cm}cp{3cm}}
      \pbox{20cm}{#1}& $\Box$    & \pbox{20cm}{#2} \\%
    \end{tabular}
  \end{center}%
}

\title{BML 5542}
\author{Erik Sjostrand}
\begin{document}
\maketitle
\tableofcontents

\section{Introduction}

Welcome to BML! This is a normal paragraph, and above we can see
the \texttt{\#+TITLE}, the \texttt{\#+AUTHOR} and the \texttt{\#+DESCRIPTION} of the file. \texttt{\#+TITLE} is
the name of the system and \texttt{\#+DESCRIPTION} is a \emph{short summary of} how
the system works. \texttt{\#+AUTHOR} is self explanatory. \texttt{Introduction} above,
``headed by an asterisk'', sets a section at the first level (the second
level would have two asterisks etc).
\bigbreak
In the paragraph above I encapsulated some words between equal
signs. This means that they will show up as a monospaced font when
exported to HTML or LaTeX. It is also possible to make words (or
sentences) \textbf{strong or bold} or \emph{in italics}.
\bigbreak
The system presented in this example file is meant to show many
of the current features in BML. Let's start with the basic opening
structure of the system:
\bigbreak
\
\dirtree{%%
 .1 \O. 
 .2 1\BC......\begin{minipage}[t]{0.8\textwidth}
2+\BC {.} Natural or balanced
\end{minipage}. 
 .2 1\BD......\begin{minipage}[t]{0.8\textwidth}
4+ suit, unbalanced
\end{minipage}. 
 .2 1M......\begin{minipage}[t]{0.8\textwidth}
5+ suit
\end{minipage}. 
 .2 2\BC......\begin{minipage}[t]{0.8\textwidth}
20--21 bal / Any game force
\end{minipage}. 
 .2 2\BD......\begin{minipage}[t]{0.8\textwidth}
6+\BH\ or 6+\BS , 5--9 hcp
\end{minipage}. 
 .2 2\BH\BS.....\begin{minipage}[t]{0.8\textwidth}
6+ suit, 10--13 hcp
\end{minipage}. 
 .2 2NT.....\begin{minipage}[t]{0.8\textwidth}
22--24
\end{minipage}. 
 .2 3X......\begin{minipage}[t]{0.8\textwidth}
Preemptive
\end{minipage}. 
 .2 3NT.....\begin{minipage}[t]{0.8\textwidth}
Gambling
\end{minipage}. 
}
\bigbreak
The above is an example of a bidding table; the reason why BML is
more suited for bridge system notes than other markup languages. You
start by writing the bid, then a number of whitespaces, and then the
description of the bid. Simple! C is for clubs, D for diamonds, H
for hearts, S for spades and N for no trump. There are also some
special cases which you could use, above we use 1M (1H and 1S), 2HS
(2H and 2S) and 3X (3C, 3D, 3H and 3S). We'll see more of these
later.
\bigbreak
The reason why the 1NT opening is left out above is a secret for
now!
\bigbreak
\section{The 1\pdfc\ opening}

You might have noticed the \BC\ in the title of this section? This
will be replaced by a club suit symbol when exported. The same is
true for \BD , \BH\ and \BS\ (but these will be converted to diamonds,
hearts and spades, of course).
\bigbreak
In this example we use transfer responses to the 1\BC\ opening:
\bigbreak
\
\dirtree{%%
 .1 1\BC--. 
 .2 1red....\begin{minipage}[t]{0.8\textwidth}
Transfer{.} 4+ major, 0+ hcp
\end{minipage}. 
 .2 1\BS......\begin{minipage}[t]{0.8\textwidth}
INV+ with 5+\BD\ / Negative NT
\end{minipage}. 
 .2 1NT.....\begin{minipage}[t]{0.8\textwidth}
Game forcing, 5+\BC\ or balanced
\end{minipage}. 
 .2 2\BC......\begin{minipage}[t]{0.8\textwidth}
5+\BC , 5--9 hcp
\end{minipage}. 
 .2 2X......\begin{minipage}[t]{0.8\textwidth}
6+ suit, 4--8 hcp
\end{minipage}. 
 .2 2NT.....\begin{minipage}[t]{0.8\textwidth}
Invitational
\end{minipage}. 
}
\bigbreak
By writing 1C--- we define that the following bids should be
continuations to the sequence 1C. We could write 1C- or 1C-- too,
the number of dashes only matters to the way the output looks. Also
note the 1red response, this defines both 1D and 1H.
\bigbreak
\subsection{After a transfer}

This section has two asterisks, meaning it will be at level two
(so its a subsection). You might also have noticed that the
paragraphs, the sections and the bidtables are separated by a
blank line. This is important in BML, as the blank line is used to
separate elements.
\bigbreak
\
\dirtree{%%
 .1 1\BC-1\BD. 
 .2 1\BH......\begin{minipage}[t]{0.8\textwidth}
Minimum with 2--3\BH 
\end{minipage}. 
 .3 1\BS......\begin{minipage}[t]{0.8\textwidth}
4+\BH , 4\BS , at most invitational
\end{minipage}. 
 .3 1NT.....\begin{minipage}[t]{0.8\textwidth}
Sign off
\end{minipage}. 
 .3 2\BC......\begin{minipage}[t]{0.8\textwidth}
Puppet to 2\BD 
\end{minipage}. 
 .4 2\BD......\begin{minipage}[t]{0.8\textwidth}
Forced
\end{minipage}. 
 .5 2\BH......\begin{minipage}[t]{0.8\textwidth}
Mildly invitational with 5\BH 
\end{minipage}. 
 .5 2\BS......\begin{minipage}[t]{0.8\textwidth}
Invitational, 5+\BH\ and 4\BS 
\end{minipage}. 
 .5 2NT.....\begin{minipage}[t]{0.8\textwidth}
Strongly invitational with 5\BH 
\end{minipage}. 
 .5 3\BC......\begin{minipage}[t]{0.8\textwidth}
Invitational with 4\BH\ and 5+\BC 
\end{minipage}. 
 .5 3\BD......\begin{minipage}[t]{0.8\textwidth}
Invitational with 4\BH\ and 5+\BD 
\end{minipage}. 
 .5 3\BH......\begin{minipage}[t]{0.8\textwidth}
6\BH , about 11--12 hcp
\end{minipage}. 
 .3 2\BD......\begin{minipage}[t]{0.8\textwidth}
Artificial game force
\end{minipage}. 
 .3 2\BH......\begin{minipage}[t]{0.8\textwidth}
6+\BH , about 9--10 hcp
\end{minipage}. 
 .2 1\BS......\begin{minipage}[t]{0.8\textwidth}
5+\BC , 4+\BS , unlimited
\end{minipage}. 
 .2 1NT.....\begin{minipage}[t]{0.8\textwidth}
17--19 bal, 2--3\BH 
\end{minipage}. 
 .2 2\BC......\begin{minipage}[t]{0.8\textwidth}
5+\BC , unbal, 0--2\BH , 0--3\BS 
\end{minipage}. 
 .2 2\BD......\begin{minipage}[t]{0.8\textwidth}
Reverse
\end{minipage}. 
 .2 2\BH......\begin{minipage}[t]{0.8\textwidth}
Minimum, 4\BH 
\end{minipage}. 
 .2 2\BS......\begin{minipage}[t]{0.8\textwidth}
16+ hcp, 5+\BC\ and 4+\BH 
\end{minipage}. 
 .3 3\BD......\begin{minipage}[t]{0.8\textwidth}
Retransfer
\end{minipage}. 
 .4 3\BH. 
 .5 3\BS......\begin{minipage}[t]{0.8\textwidth}
Cue bid, slam interest
\end{minipage}. 
 .5 4\BC\BD.....\begin{minipage}[t]{0.8\textwidth}
Cue bid, slam interest
\end{minipage}. 
 .5 4\BH......\begin{minipage}[t]{0.8\textwidth}
To play
\end{minipage}. 
 .3 3\BH......\begin{minipage}[t]{0.8\textwidth}
Invitational
\end{minipage}. 
 .3 3\BS......\begin{minipage}[t]{0.8\textwidth}
Splinter
\end{minipage}. 
 .3 4\BC\BD.....\begin{minipage}[t]{0.8\textwidth}
Splinter
\end{minipage}. 
 .3 4\BH......\begin{minipage}[t]{0.8\textwidth}
To play
\end{minipage}. 
 .2 2NT.....\begin{minipage}[t]{0.8\textwidth}
16+ hcp, 6+\BC {.} 18+ if 3\BH 
\end{minipage}. 
 .3 3\BC......\begin{minipage}[t]{0.8\textwidth}
Suggestion to play
\end{minipage}. 
 .3 3\BD......\begin{minipage}[t]{0.8\textwidth}
Relay
\end{minipage}. 
 .4 3\BH......\begin{minipage}[t]{0.8\textwidth}
3\BH , 18+ hcp
\end{minipage}. 
 .3 3\BH......\begin{minipage}[t]{0.8\textwidth}
Game forcing with 6+\BH 
\end{minipage}. 
 .2 3\BC......\begin{minipage}[t]{0.8\textwidth}
15--17 hcp, 6+\BC\ and 3\BH 
\end{minipage}. 
 .3 3\BD......\begin{minipage}[t]{0.8\textwidth}
Retransfer
\end{minipage}. 
 .3 3\BH......\begin{minipage}[t]{0.8\textwidth}
Invitational
\end{minipage}. 
 .2 3\BD......\begin{minipage}[t]{0.8\textwidth}
17--19 bal, 4\BH 
\end{minipage}. 
 .3 3\BH......\begin{minipage}[t]{0.8\textwidth}
To play
\end{minipage}. 
 .2 3\BH......\begin{minipage}[t]{0.8\textwidth}
13--15 hcp, good hand, 5+\BC\ and 4\BH 
\end{minipage}. 
 .3 3NT.....\begin{minipage}[t]{0.8\textwidth}
Asking for singleton
\end{minipage}. 
}
\bigbreak
This bidding table shows a couple of new features. The most
prominent is the ability to add continuations directly in the
table, by using whitespaces. We also see another example of
appending bids to an existing sequence, by using 1C-1D; in the
beginning.
\bigbreak
\section{Defense to 1NT}

Defining bidding when both sides bid is a little bit more tricky,
since you have to write all the bids (even passes). The opponents'
bid are indicated by encapsulating them in parentheses. P is used
for Pass, D for Double and R for Redouble.
\bigbreak
\
\dirtree{%%
 .1 (1NT)--. 
 .2 Dbl.....\begin{minipage}[t]{0.8\textwidth}
Strength, ca 15+
\end{minipage}. 
 .2 2\BC......\begin{minipage}[t]{0.8\textwidth}
At least 5-4 majors
\end{minipage}. 
 .3 (D). 
 .4 Pass....\begin{minipage}[t]{0.8\textwidth}
5+\BC , suggestion to play
\end{minipage}. 
 .4 Rdbl....\begin{minipage}[t]{0.8\textwidth}
Asking for better/longer major
\end{minipage}. 
 .4 2\BD......\begin{minipage}[t]{0.8\textwidth}
5+\BD , suggestion to play
\end{minipage}. 
 .3 (P). 
 .4 2\BD......\begin{minipage}[t]{0.8\textwidth}
Asking for better/longer major
\end{minipage}. 
 .2 2\BD......\begin{minipage}[t]{0.8\textwidth}
A weak major or a strong minor
\end{minipage}. 
 .3 (P). 
 .4 2\BH......\begin{minipage}[t]{0.8\textwidth}
Pass/correct
\end{minipage}. 
 .4 2\BS......\begin{minipage}[t]{0.8\textwidth}
Pass/correct
\end{minipage}. 
 .4 2NT.....\begin{minipage}[t]{0.8\textwidth}
Asking
\end{minipage}. 
 .2 2\BH\BS.....\begin{minipage}[t]{0.8\textwidth}
Constructive
\end{minipage}. 
 .2 2NT.....\begin{minipage}[t]{0.8\textwidth}
5-5 minors
\end{minipage}. 
 .2 3X......\begin{minipage}[t]{0.8\textwidth}
Preemptive
\end{minipage}. 
}
\bigbreak
Note that the above is only for a direct overcall over 1NT. To
define the above also when balancing we will use BML's
copy/cut/paste functionality in order to show that you do not
have to write it all over again. Take a look below (only visible in
the \emph{example.txt} or generated Full Disclosure file, not in HTML or LaTeX):
\bigbreak
First we used the \texttt{\#COPY} command; the text between \texttt{\#COPY} and \texttt{\#ENDCOPY}
got put into a sort of clipboard, with the tag nt\_defense which we
specified. To paste it into the bidding table above we used
the \texttt{\#PASTE} command. We also used the \texttt{\#HIDE} option. When this is
present in a bidding table the table will be exported to Full
Disclosure, but not to HTML or LaTeX.
\bigbreak
You could also include other BML-files into your main file by using
the \texttt{\#INCLUDE} command. Just type \texttt{\#INCLUDE <filename>} and the entire
file will be inserted at the point where you wrote the command. This
is a useful way to separate your system into modules, or perhaps
just to make it more manageable.
\bigbreak
It is also possible to add continuations when the opponents
interfere:
\bigbreak
\
\dirtree{%%
 .1 1\BC-(1\BD)--. 
 .2 Dbl.....\begin{minipage}[t]{0.8\textwidth}
4+\BH 
\end{minipage}. 
 .2 1\BH......\begin{minipage}[t]{0.8\textwidth}
4+\BS 
\end{minipage}. 
 .2 1\BS......\begin{minipage}[t]{0.8\textwidth}
INV+ with 5+\BD\ / Negative NT
\end{minipage}. 
 .2 1NT.....\begin{minipage}[t]{0.8\textwidth}
Game forcing, 5+\BC\ or balanced
\end{minipage}. 
 .2 2\BC......\begin{minipage}[t]{0.8\textwidth}
5+\BC , 5--9 hcp
\end{minipage}. 
 .2 2X......\begin{minipage}[t]{0.8\textwidth}
6+ suit, 4--8 hcp
\end{minipage}. 
 .2 2NT.....\begin{minipage}[t]{0.8\textwidth}
Invitational
\end{minipage}. 
}
\bigbreak
\section{The 1NT opening}

Here's the reason why I left out the 1NT opening earlier: I will
show how to make sequences dependant on vulnerability and
seat. This will be a bit messy, so hold tight!
\bigbreak
We start by cutting our NT-module, since this will be used on all
NT-openings. \texttt{\#CUT} is similar to the \texttt{\#COPY} command, but using \texttt{\#CUT}
means that it isn't parsed as a bidding table until it is pasted.
\bigbreak
The \texttt{\#VUL} command is used to set the vulnerability. It takes an
argument of two characters, each can be Y, N or 0. The first
character asks if we are vulnerable and the second asks if our
opponents are vulnerable. Y is for Yes, N is for No and 0 means that
it doesn't matter.
\bigbreak
The \texttt{\#SEAT} command sets the seat in which the bid should be valid. 0
means that the seat doesn't matter (all seats), 12 means first or
second and 34 means third or fourth. 1--4 could also be used.
\bigbreak
So when we're not vulnerable we open 1NT 12--14 in 1st and 2nd seat.
\bigbreak
But in third and fourth seat it is 14--16.
\bigbreak
When we're vulnerable we always open 1NT 14--16.
\bigbreak
\
\dirtree{%%
 .1 1NT--. 
 .2 2\BC......\begin{minipage}[t]{0.8\textwidth}
Stayman
\end{minipage}. 
 .3 2\BD......\begin{minipage}[t]{0.8\textwidth}
No major
\end{minipage}. 
 .3 2NT.....\begin{minipage}[t]{0.8\textwidth}
4-4 majors, minimum
\end{minipage}. 
 .3 3\BC......\begin{minipage}[t]{0.8\textwidth}
4-4 majors, maximum
\end{minipage}. 
 .2 2red....\begin{minipage}[t]{0.8\textwidth}
Transfer
\end{minipage}. 
 .2 2\BS......\begin{minipage}[t]{0.8\textwidth}
Minor suit stayman
\end{minipage}. 
 .2 2NT.....\begin{minipage}[t]{0.8\textwidth}
Invitational
\end{minipage}. 
}
\bigbreak
We've been using the \texttt{\#HIDE} command, so we don't have to see our
NT-system over and over again. This time tough we paste it
normally, so that we see it at least once.
\bigbreak
\section{Lists}

I'd like to show you how to make lists in BML. It is pretty
simple:
\bigbreak
\begin{itemize}
\item Here's a list!

\item With a couple of

\item Items in it

\end{itemize}
\bigbreak
You could also make ordered lists:
\bigbreak
\begin{enumerate}
\item This is ordered

\item Just add numbers

\item To each item

\end{enumerate}
\bigbreak
\end{document}
